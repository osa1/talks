\documentclass{beamer}

\usepackage[utf8]{inputenc}
\usepackage{tikz}
\usepackage{amssymb}
\usepackage{minted}
\usepackage{natbib}

\setbeamertemplate{itemize items}[square]

\begin{document}
\title{Rebooting Supercompilation for Haskell}

\author[Ömer\,S.\,Ağacan \& Ryan\,R.\,Newton]
{%
  \texorpdfstring{
    \begin{columns}
      \column{.45\linewidth}
      \centering
      Ömer S. Ağacan\\
      \href{mailto:oagacan@indiana.edu}{oagacan@indiana.edu}
      \column{.45\linewidth}
      \centering
      Ryan R. Newton\\
      \href{mailto:rrnewton@indiana.edu}{rrnewton@indiana.edu}
    \end{columns}
  }
  {Ömer\,S.\,Ağacan \& Ryan\,R.\,Newton}
}

\date{\today}

\frame{\titlepage}

%\frame{\frametitle{Table of contents}\tableofcontents}

\begin{frame}
    \frametitle{Rebooting Supercompilation for Haskell}

    \begin{itemize}[<+->]
        \item An overview of supercompilation and problems associated with it.
        \item Why it's worth rebooting, and why GHC is a great compiler to base
            this work on.
        \item My preliminary work, and problems I encountered while working on
            a GHC plugin.
        \item What's next? My research goals.
    \end{itemize}
\end{frame}

\begin{frame}

    \frametitle{Supercompilation: An overview}

    \begin{itemize}[<+->]
        \item
            The paper that describes the idea in English: "The Concept of a
            Supercompiler" \citet{Turchin86theconcept}.
        \item
            High-level idea:
            \setbeamertemplate{itemize items}[circle]
            \begin{itemize}
                \item
                    Evaluate programs in compile-time, while making the most out
                    of known inputs and definitions.
                    \setbeamertemplate{itemize items}[square]
                    \begin{itemize}
                        \item
                            Definitions of used functions.
                        \item
                            Statically known arguments of functions.
                        \item
                            When branching, propagate learned information
                            through branches and make use of that information
                            while compiling branches.
                    \end{itemize}
            \end{itemize}
    \end{itemize}

\end{frame}

\begin{frame}

    \frametitle{Supercompilation: An overview}

    \begin{itemize}
        \item
            The paper that describes the idea in English: "The Concept of a
            Supercompiler" \citet{Turchin86theconcept}.
        \item
            High-level idea: (contd)
            \setbeamertemplate{itemize items}[circle]
            \begin{itemize}
                \item
                    Evaluate programs in compile-time, while making the most out
                    of known inputs and definitions.
                    \setbeamertemplate{itemize items}[square]
                    \begin{itemize}[<+->]
                        \item
                            Most of the time the goal is to generate more
                            efficient programs. \newline
                            (but see \citet{Klyuchnikov2010proving} for a
                            different use of supercompilation)
                    \end{itemize}
            \end{itemize}
    \end{itemize}

\end{frame}

\begin{frame}

    \frametitle{Supercompilation: An overview}

    \begin{itemize}[<+->]
        \item
            One nice idea here is to base supercompilation algorithm on the
            language's operational semantics, as done in \citet{SCbyEval}.

        \item
            This optimizes in the sense that:

            If we have a programs $\mathcal{P}_1$ and $\mathcal{P}_2$, and
            \newline
            $\mathcal{P}_1 \Downarrow v$ in $N$ steps and \newline
            $\mathcal{P}_2 \Downarrow v$ in $M$ steps, \newline
            we consider $\mathcal{P}_2$ optimized if $M \textless N$.

        \item
            An approximation. It's very unlikely that all of the rules have same
            costs.
    \end{itemize}

\end{frame}

\begin{frame}[fragile]

    \frametitle{An example}

    \begin{itemize}[<+->]
        \item[]
            \begin{minted}{haskell}
mapOfMap f g = (.) (map f) (map g)
            \end{minted}

        \item[]
            We compile RHS of this definition, and we introduce a new definition
            at each step. We give it a fresh name, and add arguments for free
            variables in the expression.

        \item[]
            When we get stuck, we keep evaluating sub-expressions.

    \end{itemize}

\end{frame}

\begin{frame}[fragile]

    \frametitle{An example}

    \begin{itemize}[<+->]

        \item[]
            \begin{minted}{haskell}
mapOfMap f g = (.) (map f) (map g)
            \end{minted}

        \item[]
            Lookup \texttt{(.)}:
            $\beta$-reduction:

        \item[]
            \begin{minted}{haskell}
h1 f g = (\f1 -> \f2 -> \a -> f1 (f2 a))
           (map f) (map g)
            \end{minted}

        \item[]
            $\beta$-reduction:

        \item[]
            \begin{minted}{haskell}
h2 f g = (\f2 a -> (map f) (f2 a)) (map g)
            \end{minted}

        \item[]
            $\beta$-reduction:

        \item[]
            \begin{minted}{haskell}
h3 f g a = (map f) (map g a)
            \end{minted}

    \end{itemize}

\end{frame}

\begin{frame}[fragile]

    \begin{itemize}[<+->]

        \item[]
            \begin{minted}{haskell}
h3 f g a = (map f) (map g a)
            \end{minted}

        \item[] Lookup \texttt{map}:

        \item[]
            \begin{minted}{haskell}
h5 f g a = ((\f -> \l ->
               case l of
                 []    -> []
                 h : t -> f h : map f t) f) (map g a)
              \end{minted}

          \item[] $\beta$-reduction (twice):

          \item[]
              \begin{minted}{haskell}
h6 f g a = case (map g a) of
             []    -> []
             h : t -> f h : map f t
              \end{minted}

          \item[] Lookup \texttt{map}, beta reduction:

          \item[]
              \begin{minted}{haskell}
h7 f g a = case (case a of
                   []      -> []
                   h1 : t1 -> g h1 : map g t1) of
             []      -> []
             h0 : t0 -> f h0 : map f t0
              \end{minted}

    \end{itemize}

\end{frame}

\begin{frame}[fragile]

    \begin{itemize}[<+->]
        \item[] Case-of-case transformation:
            (\citet{Jones98atransformation-based})

        \item[]
            \begin{minted}{haskell}
h8 f g a = case a of
             [] -> case [] of
                     []      -> []
                     h0 : t0 -> f h0 : map f t0
             h1 : t1 -> case (g h1 : map g t1) of
                          []      -> []
                          h0 : t0 -> f h0 : map f t0
            \end{minted}

        \item[]
            At this point we consider all branches, let's start with first one:

        \item[]
            \begin{minted}{haskell}
            case [] of
              [] -> []
              h0 : t0 -> f h0 : map f t0
            \end{minted}

        \item[]
            Known case reduction evaluates this to it's final form, and we
            update our expression to:

    \end{itemize}

\end{frame}

\begin{frame}[fragile]

    \begin{itemize}[<+->]
        \item[]
            \begin{minted}{haskell}
h9 f g a = case a of
             []      -> []
             h1 : t1 -> case (g h1 : map g t1) of
                          []      -> []
                          h0 : t0 -> f h0 : map f t0
            \end{minted}

        \item[] Second branch:

        \item[]
            \begin{minted}{haskell}
            case (g h1 : map g t1) of
              []      -> []
              h0 : t0 -> f h0 : map f t0
            \end{minted}

        \item[]
            Since \texttt{h0} and \texttt{t0} are linear in the RHS of the
            second branch, we can just do substitution, without introducing
            lets:

        \item[]
            \begin{minted}{haskell}
            f (g h1) : map f (map g t1)
            \end{minted}

        \item[] So we now have:

        \item[]
            \begin{minted}{haskell}
h10 f g a = case a of
              []      -> []
              h1 : t1 -> f (g h1) : map f (map g t1)
            \end{minted}

    \end{itemize}

\end{frame}

\begin{frame}[fragile]

    \begin{itemize}[<+->]
        \item[]
            As in \texttt{h8}, we again consider alternatives. There's nothing
            to do in first branch.

        \item[]
            \begin{minted}{haskell}
            f (g h1) : map f (map g t1)
            \end{minted}

        \item[]
            This term is already in WHFN, but we consider sub-terms.

        \item[]
            \begin{minted}{haskell}
            f (g h1)
            \end{minted}

        \item[]
            We can't do anything about this, all names are free. We consider the
            second sub-term.

    \end{itemize}

\end{frame}

\begin{frame}[fragile]

    \begin{itemize}[<+->]

        \item[]
            \begin{minted}{haskell}
            map f (map g t1)
            \end{minted}

        \item[]
            This looks a lot like one of the terms we compiled before.  Remember
            \texttt{h3}:

        \item[]
            \begin{minted}{haskell}
h3 f g a = map f (map g a)
            \end{minted}

        \item[]
            Our current term is just a renaming of \texttt{h3}. We started with
            \texttt{h3} and came across the same term. If we continue we'll
            loop.

        \item[] Instead, we generate a call to \texttt{h3}.

        \item[]
            \begin{minted}{haskell}
            h3 f g t1
            \end{minted}

        \item[]
            Since that \texttt{h11} is optimized version of \texttt{h3}, we
            replace the call to \texttt{h3} with a call to \texttt{h11}, and we
            have our final definition:

        \item[]
            \begin{minted}{haskell}
h11 f g a = case a of
              []      -> []
              h1 : t1 -> f (g h1) : h11 f g t1
            \end{minted}

    \end{itemize}

\end{frame}

\begin{frame}

    \frametitle{Short digression}

    \begin{itemize}[<+->]

        \item[]
            This looked a lot similar to deforestation(\citet{deforestation}).
        \item[]
            Indeed, all of the steps we took here are described in the original
            deforestation paper.
        \item[]
            Some of the important differences are:
            \begin{itemize}[<+->]
                \item
                    No linearity restriction. (in the linear case they would do
                    similar things -- see \citet{towardsunifying} for a
                    comparison)
                \item We do generalization. (not demonstrated here)
            \end{itemize}

    \end{itemize}

\end{frame}

\begin{frame}

    \frametitle{Operations of a supercompiler - Driving}

    We evaluated the program, and while doing that we were careful
    with previously evaluated terms: \textcolor{blue}{Driving}.

\end{frame}

\begin{frame}[fragile]

    \frametitle{Operations of a supercompiler - Splitting}

    \begin{itemize}[<+->]

        \item[]
            When we're stuck because the expression we do pattern matching on
            couldn't take any more steps, we evaluated branches:

        \item[]
            \begin{minted}{haskell}
        case a of
          []      -> []
          h1 : t1 -> f (g h1) : map f (map g t1)
            \end{minted}

        \item[]
            When we're stuck because the term is in WHNF, we evaluated subterms:

        \item[]
            \begin{minted}{haskell}
        f (g h1) : map f (map g t1)
            \end{minted}

        \item[]
            This is called \textcolor{blue}{splitting}.

    \end{itemize}

\end{frame}

\begin{frame}[fragile]

    \frametitle{Operations of a supercompiler - matching}

    \begin{itemize}[<+->]
        \item[]
            We need to "compare" current expression with our history of
            expressions, to prevent loops, and generate optimized functions.

        \item[]
            \begin{minted}{haskell}
h3 f g a = map f (map g a)
...
h10 f g a = ... map f (map g t1) ...
            \end{minted}

        \item[]
            We also need a way to call optimized function when this happens.

        \item[]
            \textcolor{blue}{Matching}, returns all the necessary information to
            replace current expression with a function call to optimized
            function.

    \end{itemize}

\end{frame}

\begin{frame}[fragile]

    \frametitle{Operations of a supercompiler - termination checking}

    \begin{itemize}[<+->]
        \item[]
            Something we haven't demonstrated so far.

        \item[]
            \begin{minted}{haskell}
reverse_acc []      acc = acc
reverse_acc (h : t) acc = reverse_acc t (h : acc)
            \end{minted}

        \item[]
            \begin{minted}{haskell}
h0 lst = reverse_acc (reverse_acc lst []) []
            \end{minted}
    \end{itemize}
\end{frame}

\begin{frame}[fragile]
    \frametitle{Operations of a supercompiler - termination checking}
    \begin{minted}{haskell}
h0 lst = reverse_acc (reverse_acc lst []) []
...
h5 lst =
  case lst of
    []      -> []
    h1 : t1 ->
      case (reverse_acc t1 (h1 : [])) of
        []      -> []
        h0 : t0 -> reverse_acc t0 (h0 : [])
...
h_ lst = ... reverse_acc t1 (h1 : []) ...
...
h_ lst = ... reverse_acc t2 (h2 : h1 : []) ...
...
    \end{minted}
\end{frame}


\begin{frame}[fragile]
    \frametitle{Operations of a supercompiler - termination checking}
    \begin{itemize}[<+->]
        \item[]
            \begin{minted}{haskell}
reverse_acc []      acc = acc
reverse_acc (h : t) acc = reverse_acc t (h : acc)
            \end{minted}

        \item[]
            The growing argument(accumulator) makes this example tricky.
        \item[]
            We need a \textcolor{blue}{termination checker} to stop in cases
            like this.
        \item[]
            What to do after stopping like this is another story.\newline (see
            \citet{callbyneed-sc})

    \end{itemize}

\end{frame}

\begin{frame}[fragile]
    \frametitle{Operations of a supercompiler - issues}

    \begin{itemize}[<+->]
        \item[] Each one has tricky problems.
        \item[] Splitter:
            \begin{itemize}[<+->]
                \item
                    We want to propagate information to sub-terms.
                \item
                    Propagate too much: We end up duplicating work.
                \item
                    Example: (from \citet{callbyneed-sc})
                    \begin{minted}{haskell}
let n = fib 100
    b = n + 1
    c = n + 2
 in (b, c)
                    \end{minted}
            \end{itemize}
    \end{itemize}

\end{frame}

\begin{frame}[fragile]
    \frametitle{Operations of a supercompiler - issues}

    \begin{itemize}
        \item[] Each one has tricky problems.
        \item[] Splitter:
            \begin{itemize}
                \item
                    We want to propagate information to sub-terms.
                \item
                    Propagate too little: We miss optimization opportunities.
            \end{itemize}
    \end{itemize}

\end{frame}

\begin{frame}[fragile]
    \frametitle{Operations of a supercompiler - issues}

    \begin{itemize}
        \item[] Each one has tricky problems.
        \item[] Splitter:
            \begin{itemize}
                \item
                    We want to propagate information to sub-terms.
                \item
                    Propagate too little: We miss optimization opportunities.
                \item
                    Example: (from \citet{callbyneed-sc})
                    \begin{minted}{haskell}
let map = ...
    ys = map f zs
    xs = map g ys
 in Just xs
                    \end{minted}
            \end{itemize}
    \end{itemize}

\end{frame}

\begin{frame}[fragile]
    \frametitle{Operations of a supercompiler - issues}

    \begin{itemize}[<+->]
        \item[] Each one has tricky problems.
        \item[] Matcher:
            \begin{itemize}[<+->]
                \item
                    Syntactic equality(modulo $\alpha$-renaming) almost never
                    holds.
                \item
                    We may end up sharing work, which is not a good thing in
                    general(increases residency, see
                    \citet{commonsubexpression}) (example from
                    \citet{callbyneed-sc})
                \item
                    \begin{minted}{haskell}
S0 =
    let a = fib y
        b = fib y
     in (a, b)

S1 = let a = fib y in (a, a)
                    \end{minted}

            \end{itemize}
    \end{itemize}

\end{frame}

\begin{frame}
    \frametitle{Operations of a supercompiler - issues}

    \begin{itemize}[<+->]
        \item
            "We do not attempt to supercompile the full Nofib suite because the
            other Nofib benchmarks are considerably more complicated and
            generally suffer from extremely long supercompilation times."
        \item
            \citet{timeandspace} focuses on compilation performance, and
            reports \textit{$<$3 seconds} for all the small programs from nofib.
    \end{itemize}
\end{frame}

\begin{frame}
    \frametitle{Latest work}

    \begin{itemize}[<+->]
        \item
            \citet{callbyneed-sc} laid out a great framework, with nicely defined
            components. (Splitter, matcher, termination checker etc.)
        \item
            Problems and current solutions are documented nicely.
        \item
            We don't have any solutions that work well on \textit{all} programs.
        \item
            We don't have a usable implementation.
    \end{itemize}

\end{frame}

\begin{frame}
    \frametitle{It's worth rebooting!}

    \begin{itemize}[<+->]
        \item
            \citet{callbyneed-sc} shows some great potential.
        \item
            Up to $-95.1\%$ runtime improvement.
        \item
            Up to $-100.0\%$ allocation improvement.
    \end{itemize}

\end{frame}

\begin{frame}
    \frametitle{GHC Plugin API}

    \begin{itemize}[<+->]
        \item
            GHC plugin API can help here, we can implement a supercompiler as a
            plugin, get immediate feedback from users.
        \item
            No need to merge anything to GHC(in theory).
        \item
            But... GHC API feels like \textit{exposed internals} rather than an
            \textit{API}.
    \end{itemize}
\end{frame}

\begin{frame}
    \frametitle{Problems with GHC Plugin API}

    \begin{itemize}[<+->]
        \item
            No easy ways to do most basic stuff: Moving terms
            around(substitutions), known-case reduction, case-of-case, etc.
            (all done in some parts of Core-to-Core passes, need to reverse
            engineer)
        \item
            No easy way to annotate Core syntax. Duplicating the syntax means
            duplicating huge amounts of code.
        \item
            Working on Core hard: Invariants are encoded as partial functions
            without any helpful error messages -- if we're lucky, there's a
            NOTE.
        \item
            Some things are not clear. (Types are first-class, but can I use
            them wherever I want? The definition allows this)
    \end{itemize}
\end{frame}


\begin{frame}
    \frametitle{A more fundamental problem}

    \begin{itemize}[<+->]
        \item[]
            Supercompilation is inherently a whole-program optimization
            technique. ("-O99")
        \item[]
            It can optimize just a single definition, but more definitions mean
            more optimizations.
        \item[]
            GHC only saves definitions of small definitions, or definitions with
            \texttt{INLINE} or \texttt{INLINABLE}.
        \item[]
            Ideally we need all the definitions in a module in it's \texttt{.hi}
            file.
        \item[]
            Idea: Use \texttt{-fexpose-all-unfoldings} all the time, distribute
            \texttt{base} with this option.
    \end{itemize}
\end{frame}

\begin{frame}
    \frametitle{Roadmap}

    \begin{itemize}[<+->]
        \item
            Working on implementing the supercompiler described in
            \citet{callbyneed-sc}.
        \item
            (Hopefully) Improving GHC plugin API on the way.
        \item
            Collecting benchmark programs - send yours! (with expected
            optimizations)
        \item
            Once we have that, hopefully research will follow.
            \begin{itemize}[<+->]
                \item
                    Focus on specific parts(matcher, splitter etc.).
                \item
                    Work on some of the obvious improvements, like parallelizing
                    the matcher.
                \item
                    I'm open for more ideas!
            \end{itemize}
    \end{itemize}
\end{frame}

\begin{frame}[allowframebreaks]
    \frametitle{References}

    \bibliographystyle{abbrvnat}
    \bibliography{refs}
\end{frame}

\end{document}
